\message{ !name(fenwick.lhs.tex)}
\message{ !name(fenwick.lhs) !offset(1522) }
\section{Deriving Fenwick Operations} \label{sec:fenwick-ops}

We can now finally derive the required operations on Fenwick array
indices for moving through the tree, by starting with operations on a
binary indexed tree and conjugating by conversion to and from Fenwick
indices.  First, in order to fuse away the resulting conversion, we
will need a few lemmas.

\begin{lem} \label{lem:incshr}
  For all odd |bs :: Bits|,
  \begin{itemize}
  \item |(shr . dec) bs = shr bs|
  \item |(shr . inc) bs = (inc . shr) bs|
  \end{itemize}
\end{lem}
\begin{proof}
  Both are immediate by definition.
\end{proof}

\begin{lem} \label{lem:incwhile}
  The following both hold for all |Bits| values:
  \begin{itemize}
  \item |inc . while odd shr = while even shr . inc|
  \item |dec . while even shr = while odd shr . dec|
  \end{itemize}
\end{lem}
\begin{proof}
  Easy proof by induction on |Bits|.  For example, for the |inc| case,
  the functions on both sides discard consecutive 1 bits and then flip
  the first 0 bit to a 1.
\end{proof}

Finally, we will need one somewhat technical lemma related to shifting
zero bits in and out of the right side of a value.

\begin{lem} \label{lem:shlshr}
  For all positive |Bits| values less than $2^{n+2}$,
  \[ |while (not . test (n+1)) shl . while even shr = while (not . test (n+1)) shl|. \]
\end{lem}
\begin{proof}
  Intuitively, this says that if we first shift out all the zero bits
  and then left shift until bit $n+1$ is set, we could get the same
  result by forgetting about the right shifts entirely; shifting out
  zero bits and then shifting them back in should be the identity.

  The proof is by induction.  If |x = xs :. I| is odd, the equality is
  immediate since |while even shr x = x|. Otherwise, if |x = xs :. O|,
  on the left-hand side we have
  \begin{sproof}
    \stmt{|(while (not . test (n+1)) shl . while even shr) (xs :. O)|}
    \reason{=}{Definition of |while|}
    \stmt{|(while (not . test (n+1)) shl . while even shr) xs|}
  \end{sproof}
  whereas on the right-hand side,
  \begin{sproof}
    \stmt{|while (not . test (n+1)) shl (xs :. O)|}
    \reason{=}{Definition of |shl|}
    \stmt{|while (not . test (n+1)) shl (shl xs)|}
    \reason{=}{Definition of |while|; $|xs| < 2^{n+1}$ so |test (n+1) xs = False| }
    \stmt{|while (not . test (n+1)) shl xs|}
  \end{sproof}
  These are equal by the induction hypothesis.
\end{proof}

With these lemmas under our belt, let's see how to move around a
Fenwick array in order to implement |update| and |query|; we'll begin
with |update|. When implementing the |update| operation, we need to start at a leaf
and follow the path up to the root, updating all the active nodes
along the way.  In fact, for any given leaf, its closest active parent
is precisely the node stored in the slot that used to correspond to
that leaf (see \pref{fig:right-leaning}).  So to update index $i$, we
just need to start at index $i$ in the Fenwick array, and then
repeatedly find the closest active parent, updating as we go.  Recall
that the imperative code for |update| works this way, apparently
finding the closest active parent at each step by adding the |lsb| of
the current index:
\inputminted[fontsize=\footnotesize,firstline=8,lastline=10]{java}{FenwickTree.java}
\noindent
Let's see how to derive this behavior.

To find the closest active parent of a node under a binary indexing
scheme, we first move up to the immediate parent (by dividing the
index by two, \ie performing a right bit shift); then continue moving
up to the next immediate parent as long as the current node is a right
child (\ie has an odd index).  XXX picture?  This yields the definition:

\begin{code}

activeParentBinary :: Bits -> Bits
activeParentBinary = while odd shr . shr

\end{code}

This is why we used the slightly strange indexing scheme with the root
having index $2$---otherwise this definition would not work for any
node whose active parent is the root!

Now, to derive the corresponding operation on Fenwick indices, we
conjugate by conversion to and from Fenwick indices, and compute as
follows.  To make the computation easier to read, the portion being
rewritten is underlined at each step.

\begin{sproof}
  \stmt{|b2f' n . activeParentBinary . f2b' n|}
  \reason{=}{expand definitions}
  \stmt{|unshift (n+1) . ul(inc . while odd shr) . shr . dec . shift (n+1)|}
  \reason{=}{\pref{lem:incwhile}}
  \stmt{|unshift (n+1) . while even shr . inc . ul(shr . dec) . shift (n+1)|}
  \reason{=}{\pref{lem:incshr}; the output of |shift (n+1)|
    is always odd}
  \stmt{|unshift (n+1) . while even shr . ul(inc . shr) . shift (n+1)|}
  \reason{=}{\pref{lem:incshr}}
  \stmt{|unshift (n+1) . ul(while even shr . shr) . inc . shift (n+1)|}
  \reason{=}{|while even shr . shr = while even shr| on an even input}
  \stmt{|ul(unshift (n+1)) . while even shr . inc . shift (n+1)|}
  \reason{=}{Definition of |unshift|}
  \stmt{|clear (n+1) . ul(while (not . test (n+1)) shl . while even shr) . inc . ul(shift (n+1))|}
  \reason{=}{\pref{lem:shlshr}; definition of |shift|}
  \stmt{|clear (n+1) . while (not . test (n+1)) shl . inc . while even shr . set (n+1)|}
\end{sproof}
In the final step, since the input $x$ satisfies $x \leq 2^n$, we have
$|inc . shift (n+1)| < 2^{n+2}$, so \pref{lem:shlshr} applies.

Reading from right to left, the pipeline we have just computed
performs the following steps:
\begin{enumerate}
\item Set bit $n+1$ to 1
\item Shift out consecutive zeros until finding the least significant $1$ bit
\item Increment
\item Shift zeros back in to bring the most significant bit back to position $n+1$,
  then clear it.
\end{enumerate}



\begin{figure}
\begin{center}
\begin{diagram}[width=300]
\end{diagram}
\end{center}
\end{figure}

Intuitively, this does look a lot like adding the LSB!  In general, to
find the LSB, one must shift through consecutive $0$ bits until
finding the first $1$; the question is how to keep track of how many
$0$ bits were shifted on the way.  The |lsb| function itself keeps
track via the recursion stack; after finding the first $1$ bit, the
recursion stack unwinds and re-snocs all the $0$ bits recursed through
on the way.  The above pipeline represents an alternative approach:
set bit $n+1$ as a ``placeholder'' to keep track of how much we have
shifted; right shift until the first $1$ is literally in the ones
place, at which point we increment; and then shift all the $0$ bits
back in by doing left shifts until the placeholder bit gets back to
the $n+1$ place. Of course, this only works for values that are
sufficiently small that the placeholder bit will not be disturbed
throughout the operation.

To make this more formal, we begin by defining a helper function
|atLSB|, which does an operation ``at the LSB'', that is, it shifts
out 0 bits until finding a 1, applies the given function, then
restores the 0 bits:

\begin{code}

atLSB :: (Bits -> Bits) -> Bits -> Bits
atLSB _ (Rep O) = Rep O
atLSB f (bs :. O) = atLSB f bs :. O
atLSB f bs = f bs

\end{code}

\begin{lem} \label{lem:addlsb}
  For all |x :: Bits|, |x + lsb x = atLSB inc x| and |x - lsb x =
  atLSB dec x|.
\end{lem}

\begin{proof}
  Straightforward induction on $x$.
\end{proof}

We can formally relate the ``shifting with a placeholder'' scheme to
the use of |atLSB|, with the following (admittedly rather technical)
lemma:

\begin{lem} \label{lem:placeholder-scheme} Let $n \geq 1$ and let |f
  :: Bits -> Bits| be a function such that
  \begin{enumerate}
  \item |(f . set (n+1)) x = (set (n+1) . f) x| for any $0 < x \leq
    2^n$, and
  \item $|f x| < 2^{n+1}$ for any $0 < x \leq 2^n + 2^{n-1}$ as long
    as $n \geq 2$.
  \end{enumerate}
  Then for all $0 < x \leq 2^n$,
  \[ |(unshift (n+1) . f . shift (n+1)) x = atLSB f x|. \]
\end{lem}

The proof is rather tedious and not all that illuminating, so we omit
it.  However, we do note that both |inc| and |dec| fit the criteria
for |f|: incrementing or decrementing some $0 < x \leq 2^n$ cannot affect
the $(n+1)$st bit as long as $n \geq 1$, and the result of
incrementing or decrementing a number up to $2^n + 2^{n-1}$ will
certainly result in number less than $2^{n+1}$, as long as $n \geq 2$
(if $n=1$ then in fact $|inc| (2^n + 2^{n-1}) = 2^{n+1}$).  We can now
put all the pieces together show that adding the LSB at each step is
the correct way to implement |update|.

\begin{thm}
  Adding the LSB is the correct way to move up a Fenwick-indexed tree
  to the nearest active parent, that is,
  \[ |b2f' n . activeParentBinary . f2b' n = \x -> x + lsb x|. \]
\end{thm}
\begin{proof}
\begin{sproof}
  \stmt{|b2f' n . activeParentBinary . f2b' n|}
  \reason{=}{Previous calculation}
  \stmt{|unshift (n+1) . inc . shift (n+1)|}
  \reason{=}{\pref{lem:placeholder-scheme}}
  \stmt{|atLSB inc|}
  \reason{=}{\pref{lem:addlsb}}
  \stmt{|\x -> x + lsb x|}
\end{sproof}
\vspace{-3\baselineskip}
\end{proof}

We can do a similar process to derive an implementation for prefix
query (which suppsedly involves \emph{subtracting} the LSB).  Again, if we
want to compute the sum of $[1, j]$, we can start at index $j$ in the
Fenwick array, which stores the sum of the unique segment ending at
$j$.  If the node at index $j$ stores the segment $[i,j]$, we next
need to find the unique node storing a segment that ends at $i-1$.  We
can do this repeatedly, adding up segments as we go.

\begin{figure}
\begin{center}
\begin{diagram}[width=300]
  import FenwickDiagrams
  import SegTree
  import Data.Monoid
  import Control.Arrow ((***), second)

  dia :: Diagram B
  dia = vsep 0.7
    [ sampleArray
      # mkSegTree
      # rq' i j
      # fst
      # drawSegTree opts
    , (fst (leafX i n) ^& 0) ~~ (snd (leafX j n) ^& 0)
      # lc green
      # applyStyle defRangeStyle
    ]
      <> (arrowBetween' arrOpts (5 ^& (-2)) (0 ^& (-2))) # lw veryThick
      <> (arrowBetween' arrOpts (3.5 ^& (-6)) (1.5 ^& (-6))) # lw veryThick
    where
      i = 1
      j = 11
      n = length sampleArray

      de (_, (Recurse, _)) x _ y
        || location x ^. _x > location y ^. _x =
             beneath (arrowBetween' arrOpts (location y) (location x) # lw veryThick
                      <> (location x ~~ location y))
      de _ x _ y = beneath (location x ~~ location y)

      arrOpts = with & gaps .~ local 0.5

      opts :: SegTreeOpts (Visit, Sum Int) B
      opts = (mkSTOpts (showRangeOpts' False False) :: SegTreeOpts (Visit, Sum Int) B)
        { drawEdge = de
        }
\end{diagram}
\end{center}
\caption{Moving up a segment tree to find successive prefix
  segments} \label{fig:segment-tree-prefix-query-up}
\end{figure}

      % drawSTOpts :: SegTreeOpts (Visit, Sum Int) B
      % drawSTOpts = STOpts
      %   { drawNode = drawNode' (undefined :: DrawNodeOpts (Visit, Sum Int) B)
      %         -- SegNode (Visit, Sum Int) -> Int -> Int -> QDiagram B V2 Double Any
      %   , drawEdge = drawEdgeDef
      %   , stVSep = 1
      %   , leanRight = False
      %   }


Staring at \pref{fig:segment-tree-prefix-query-up} for inspiration, we
can see that what we want to do is find the \emph{left sibling} of our
\emph{closest inactive parent}, that is, we go up until finding the
first ancestor which is a right child, then go to its left sibling.
Under a binary indexing scheme, this can be implemented simply as:

\begin{code}

prevSegmentBinary :: Bits -> Bits
prevSegmentBinary = dec . while even shr

\end{code}

\begin{thm}
  Subtracting the LSB is the correct way to move up a Fenwick-indexed
  tree to the to active node covering the segment previous to the
  current one, that is,
  \[ |b2f' n . prevSegmentBinary . f2b' n = \x -> x - lsb x|. \]
\end{thm}
\begin{proof}
\begin{sproof}
  \stmt{|b2f' n . prevSegmentBinary . f2b' n|}
  \reason{=}{expand definitions}
  \stmt{|unshift (n+1) . ul(inc . dec) . while even shr . dec . shift (n+1)|}
  \reason{=}{|inc . dec = id|}
  \stmt{|ul(unshift (n+1)) . while even shr . dec . shift (n+1)|}
  \reason{=}{Definition of |unshift|}
  \stmt{|clear (n+1) . ul(while (not . test (n+1)) shl . while even shr) . dec . shift (n+1)|}
  \reason{=}{\pref{lem:shlshr}}
  \stmt{|ul(clear (n+1) . while (not . test (n+1)) shl) . dec . shift
    (n+1)|}
  \reason{=}{Definition of |unshift|}
  \stmt{|unshift (n+1) . dec . shift (n+1)|}
  \reason{=}{\pref{lem:placeholder-scheme}}
  \stmt{|atLSB dec|}
  \reason{=}{\pref{lem:addlsb}}
  \stmt{|\x -> x - lsb x|}
\end{sproof}
\vspace{-3\baselineskip}
\end{proof}


\message{ !name(fenwick.lhs.tex) !offset(-331) }
